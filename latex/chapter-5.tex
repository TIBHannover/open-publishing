\documentclass{article}

\usepackage{caption}
                
\usepackage[backend=biber,hyperref=false,citestyle=authoryear,bibstyle=authoryear]{biblatex}
                
\bibliography{bibliography}
            
\usepackage{graphicx}
                
\usepackage{calc}
                
\newlength{\imgwidth}
                
\newcommand\scaledgraphics[2]{%
                
\settowidth{\imgwidth}{\includegraphics{#1}}%
                
\setlength{\imgwidth}{\minof{\imgwidth}{#2\textwidth}}%
                
\includegraphics[width=\imgwidth,height=\textheight,keepaspectratio]{#1}%
                
}
            
\begin{document}

\title{Wikidata - weitere Werkzeuge}

\maketitle





In diesem Kapitel werden jeweils zwei Quellen zu OpenRefine und SPARQL vorgestellt und beschrieben.

\begin{figure}
\scaledgraphics{5c3fefe0-3802-4032-a442-66cfe33e8be3.png}{0.5}
\caption*{Abbildung 1: Logo von OpenRefine}\label{F25599931}
\end{figure}





\subsection{Open School Maps}\label{H1052681}



Auf der Informationsseite von OpenSchoolMaps.ch \autocite{openschoolmapsch_daten_nodate} wird das Wichtigste über OpenRefine beschrieben und erklärt. Die Erstellung eines Projekts in OpenRefine wird anhand von Bildern, sowie die Transformation eines Datensatzes werden mithilfe von Facetten und anderen Transformationsfunktionen ausführlich erläutert.


Die Verwendung von OpenRefine zur Bereinigung bzw. Duplizierung und die Integration eines Datensatzes in einen anderen Datensatz werden auch verdeutlicht und erläutert.


Auch das Web Scraping und die Geokodierung werden festgehalten und ausführlich erklärt.


\subsection{Tutorial zu OpenRefine}\label{H8107403}



In dem Tutorial zu OpenRefine \autocite{wikimediade_openrefine_2022} von Sandra Fauconnier wird das Programm Schritt für Schritt erklärt. Das Video stammt aus der Session welches online auf der WikidataCon 2021 stattgefunden hat.





\subsection{Wikidata SPARQL }\label{H4162303}



Wikidata:SPARQL \autocite{wikidata_wikidatasparql_nodate} lässt sich in folgende Punkte unterteilen:


1.     Bevor wir beginnen 


Zu diesem Punkt versteht man über die Einführung von Leitfaden und Erklärungen zu Wikidata, SPARQL, und WDQS.


2.     Sparql Basics


Zu diesem Punkt versteht man unter dem Begriff Trippeln (eine aufgebaute Satzform). 


3.     Unsere erste Abfrage


Hier geht es um die Erklärung, wie man die korrekte WDQS-Abfrage machen könnte. 


3.1.Autovervollständigung


Es geht um die Service, die man sich beim Schreiben eines Codes vereinfachen lassen kann.


4.     Fortgeschrittene Trippelmuster


Es geht um die Muster von fortgeschrittenem Tripel.


5.     Instanzen und Klassen


In diesem Punkt versteht man unter der Bedeutung von zwei Sätzen, die mit Verben „Ist“ wendet. 


6.     Eigentschaftspfade


Hier geht es um die Erklärung von Eigenschaftspfad und gibt es Übungen dazu.


7.     Qualifikatoren


In diesem Punkt versteht man unter einem Material eines Satzes.


8.     Order und Limit


In diesem Punkt kommt die Abfragen wieder zur SPARQL und diesen Abfragen ist gemeint, dass die Ergebnisse in eine Rangfolge gebracht werden, und wir dann nur die obersten Ergebnisse dieser Rangfolge betrachten wollen.


9.     Übung


Hier handelt es sich um Übungen der Sparql


1.     Bücher von Arthur Conan Doyle


2.     Chemische Elemente


3.     Flüsse, die in den Mississippi fließen


4.     Flüsse, die in den Mississippi fließen II


10.  Optional


Hier geht um einen Befehl „Optional“. Der ist für weitere Optionen der Eigenschaften.


11.  Ausdrücke, Filter und BIND


In diesem Punkt kann man die Einführung von Basiskonzept mit Werten wie Ausdrücke, Datentypen usw. verstehen.


11.1 Datentypen 


In diesem Punkt versteht man, welche Werttypen die wichtigsten in SPARQL sind. 


11.2 Operatoren


In diesem Punkt geht um die Erklärungen von mathematischen Operatoren in Sparql wie Addieren, Subtrahieren, Multiplizieren und Dividieren von Zahlen 


11.3 FILTER


In diesem Punkt geht es um die Ergebnisse einer Klausel in SPARQL zu filtern  


11.4 Bind, Bound und If


In diesem Punkt geht es um die Erklärung von drei Funktionen, die eine möglicherweise in Schleife , True or False oft benutzt werden soll.


11.5 Coalesce


Man versteht in diesem Punkt, dass Coalesce eine Abkürzung von Bind, Bound und IF ist


12.  Gruppierung 


In diesem Punkt geht es um die Erklärung zum Vereinfachen den Datenobjekten, damit man es nicht so lange Liste braucht. Man wendet sich mit Gruppierung.


13.  Values


Hier geht es um die konkrete Erklärung und deren Beispiele von Values.


14.  Darüber Hinaus 


Ist die Endung eines Tutorials des SPARQL.





\subsection{SPARQL in Wikibooks }\label{H4372507}



Hier geht es um das Tutorial von SPARQL \autocite{wikibooks_sparql_nodate}.


1.     Introduction 


1.1  Basic


In diesem Punkt bekommt man die Erklärung und Einführung eines Sparql, man versteht unter den Beispielen.


1.2   Wiki Data Query System 


In diesem Punkt geht es um die Erklärungen von WDQS und einige Befehle, die man benutzen kann. Außerdem geht es um die Erklärung von Sparql Endpoints und dem Unterstützen Format.


1.3  Wiki Data Query System Introduction 


In diesem Punkt geht es um die komplette Anleitung, wie man einfachste Query (Grundkenntnisse) in Wikidata benutzen kann. 


1.4  Prefixes 


In diesem Punkt geht es um die Erklärung von Abkürzungen der WDQS.


1.5  Sentences


Hier geht es um die Zusammenfassung die Satzbeziehung wie Komma, eckige Klammer, Punkt und semi-colon, wie man das benutzen kann.


1.6  Triples


In diesem Punkt geht es um die Erklärung von SPO (Subjekt, Prädikat und Objekt ) in einem Satz. Dieser Methode wird häufig in RDF und WDQS benutzt.


2.     SPARQL Statements 


2.1  Select


In diesem Punkt versteht man, wie man Select Befehl benutzen kann.


2.2  Optional


In diesem Punkt geht es um die Erklärung, wie man mit Triples Satz alles abgefragt werden kann. Damit kann man mit Optional Befehl benutzen.


2.3  Filtering 


In diesem Punkt geht es um die Ergebnisse einer Klausel in SPARQL zu filtern.


2.4  UNION


In diesem Punkt versteht man, wie UNION richtig benutzt werden kann.


2.5  Services


2.5.1       Service Labels and Description 


In diesem Punkt kann man verstehen, wie man das Label oder die Beschreibung der abgefragten Entitäten mit Sprach-Fallback abruft.


2.5.2       Service Around and Box


In diesem Punkt geht es um die Erklärungen eines Diensts, wie man eine Suche der Koordinaten eines Platzes innerhalb eines bestimmten Radius des Zentrums oder innerhalb eines bestimmten Begrenzungsrahmens befinden kann


2.5.3       Service Media Wiki API


In diesem Punkt kann man verstehen, wie man den Aufruf der Mediawiki-API von SPARQL und den Empfang der Ergebnisse aus der SPARQL-Abfrage finden kann.


2.6  Modifiers


In diesem Punkt kann man verstehen, wie die fünf Modifiers wie GROUP BY, HAVING, ORDER BY, LIMIT benutzen kann.


2.7  Aggregate Function


In diesem Punkt kann man verstehen, wie die sieben Funktions wie COUNT, MIN, MAX, SUM, AVG, SAMPLE and GROUP\_CONCAT 


2.8  Bad Aggregate 


Hier geht es um die Erklärung eines Fehlers von Befehl. 


3.     Advance SPARQL Topics


Hier geht es um die Erklärung von Sparql/ Property Paths, Inverse Link, Instances and classes and Reference 


3.1  Variables 


Hier geht es um die Erklärung von Query Variabel.


3.2  Expression und Function


In diesem Punkt geht es um die Erklärung sowie deren Beispiele von einigen Befehlen.


3.3  Federated Query


In diesem Punkt hat erklärt, dass Query Fähigkeit hat, um die Abfrage aus Informationen zu beantworten.


3.4  Subqueries


In diesem Punkt versteht man, dass Sparql mit Subqueries in einer anderen Abfrage verschachteln kann.


3.5  Templates 


In diesem Punkt kann der Benutzer eine oder mehrere Variablen auswählen kann, um eine Abfrage zu ändern, ohne die SPARQL-Abfragesprache kennen zu müssen. Dementsprechend kann Innerhalb des Wikidata Query Service Query Helper der Kommentar \#TEMPLATE eine einfache Vorlage erstellen.


4.     Wikidata Represätation 


4.1  Sparql/ Wikidata Qualifiers, References and Ranks


 Hier geht es um die Anleitung sowie konkrete Erklärungen von Qualifiers, References and Ranks.


 4.2  Sparql / Wikidata Language Links and Badges


Hier geht es um die Anleitung sowie konkrete Erklärungen von Language Links xsand Badges


 4.3  Wikidata Precision, Units and coordinates 


Hier geht es um die Anleitung sowie konkrete Erklärungen von Präzision, Units und Koordinaten


4.4  Wikidata Lexicographical data


 Hier geht es um die Anleitung sowie konkrete Erklärungen von Lexicographical data.


4.5  Wikidata Commons Query Service 


Hier geht es um die Anleitung sowie konkrete Erklärungen von  Wikidata Commons Query Service


\printbibliography[title={Literaturverzeichnis}]
\end{document}
