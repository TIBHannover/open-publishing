\documentclass{article}

\usepackage{hyperref}
\usepackage{caption}
\usepackage{wrapfig}
                
\usepackage[backend=biber,hyperref=false,citestyle=authoryear,bibstyle=authoryear]{biblatex}
                
\bibliography{bibliography}
            
\usepackage{graphicx}
                
\usepackage{calc}
                
\newlength{\imgwidth}
                
\newcommand\scaledgraphics[2]{%
                
\settowidth{\imgwidth}{\includegraphics{#1}}%
                
\setlength{\imgwidth}{\minof{\imgwidth}{#2\textwidth}}%
                
\includegraphics[width=\imgwidth,height=\textheight,keepaspectratio]{#1}%
                
}
            
\begin{document}

\title{Wikidata und die OpenGLAM Community}

\maketitle





\section{Was ist OpenGLAM?}\label{H9507990}



OpenGLAM bezeichnet ein Netzwerk aus Galerien, Bibliotheken, Archiven und Museen und setzt sich für die gemeinsame Nutzung kulturellen Erbes ein. Das Wort GLAM setzt sich dabei aus den englischen Wörtern „Galleries, Libraries, Archives, Museums“ zusammen.

\begin{figure}
\scaledgraphics{ce15208f-6cbe-4d4f-a6eb-ffd2e7f8fb24.png}{1}
\caption*{Bild "OpenGLAM Logo" von der Open Knowledge foundation unter der Lizenz \href{https://creativecommons.org/licenses/by/4.0/deed.de}{CC BY 4.0} via \href{https://https://commons.wikimedia.org/wiki/File:OpenGLAM_Logo.svg}{Wikimedia Commons}}\label{F81154591}
% © 2022 Open Knowledge foundation
% CC BY 4.0: https://creativecommons.org/licenses/by/4.0/
\end{figure}





Dabei bietet vor allem das Internet eine neue Möglichkeit, um ein großes Publikum anzusprechen. Sammlungen werden dadurch auffindbarer und können einfacher miteinander vernetzt werden. Die Nutzer profitieren, indem sie etwas zu den Sammlungen beitragen, sich an Ihnen beteiligen oder diese teilen.


Um digitale Inhalte oder Daten offen zu gestalten gibt es eine Definition was unter offene Daten zählt. Diese sogenannte Open Definition lässt sich wie folgt zusammenfassen:

\begin{quote}



"Daten oder Inhalte werden erst dann als offen bezeichnet, wenn es jedem freisteht, diese zu verwenden, wiederzuverwenden und weiterzuverbreiten. Dabei darf maximal die Bedingung bestehen das der Urheber namentlich genannt werden soll und/oder dass, dass Werk unter denselben Bedingungen verfügbar gemacht wird."


\end{quote}


An erster Stelle, um offene Daten zu schaffen, steht damit immer die richtige Nutzung der Lizenzen. \autocite{noauthor_openglam_nodate}


\section{Der OpenGLAM Hackathon}\label{H2719449}


\begin{wrapfigure}{l}{0.5\textwidth}
\scaledgraphics{7b6890a5-883b-4d11-9a51-d9548403ca3f.jpg}{0.5}
\caption*{Foto "First Swiss Open Cultural Data Hackathon" von OpenGLAM.ch unter der Lizenz \href{https://https://creativecommons.org/licenses/by/4.0/deed.de}{CC BY 4.0} via \href{https://https://glam.opendata.ch/hackathons/}{OpenGLAM.ch}}\label{F31606501}
% © 2022 OpenGLAM.ch
% CC BY 4.0: https://creativecommons.org/licenses/by/4.0/
\end{wrapfigure}


Die OpenGLAM Working Group innerhalb von Opendata.ch, dem Schweizer Chapter der Open Knowledge Foundation Initiative, richtet jährlich einen OpenGLAM Hackathon aus. \autocite{regenscheit_hackathons_2015}


Bei dem Hackathon werden neben kostenlosen Einführungskursen zu WikiData, in englischer und deutscher Sprache, auch vertiefende Angebote zu WikiData for Glam angeboten. Die Zielgruppe sind Datenanbieter, Softwareentwickler, Digitale Geisteswissenschaftler, Künstler, Wikimedianer und andere Interessierte, die einmal im Jahr zusammenkommen, um kulturelle Daten und Inhalte für Forschungszwecke in den Kontext von Wikipedia zu bringen.


Die Veranstaltungsserie dient zum einen dazu, dass sich die Teilnehmenden vernetzen und austauschen können und zum anderen sollen offene Datensätze und offene Sammlungen aus dem Bereich des Kulturerbes für die Weiterverwendung zur Verfügung gestellt werden.


Der Hackathon ermutigt damit, vor allem Schweizer Kulturerbe-Institutionen, ihre Daten und Inhalte für die Weiterverwendung zu öffnen. \autocite{andrea_wikidata_2021}


\section{Open Knowledge Foundation}\label{H3830330}



Auch die Open Knowledge Foundation hatte bereits 2017 mit Einführungskursen zu Wikidata und dessen Potential für Kulturinstitutionen gestartet . Im Rahmen des Workshops wurde der Aufbau und die Funtionsweise von Wikidata vermittelt und die freiwillige Arbeit der Community vorgestellt. 


Mittlerweile hat das Institut für Kunst- und Bildgeschichte (IKB) der HU Berlin von den über 50.000 Datensätzen der Glasdiasammlung mehrere tausende mit Wikidata Items verknüpft. Dabei fungiert Wikidata als Ankerpunkt für die Verbindung der eigenen Daten mit weiteren Linked Data, wie die Lokalisierung der abgebildeten Kunstwerke. \autocite{hahn_gentle_2017}


\section{WikiProject Cultural Heritage}\label{H1258277}



Innerhalb der bereits aktiven Wikidata-Community hat die Gruppe WikiProject Cultural Heritage die Vision "Wikidata als zentralen Knotenpunkt für Datenintegration, Datenanreicherung und Datenmanagement im Bereich des Kulturerbes zu etablieren". 


Die Projektgruppe wurde im Herbst 2016 gegründet und basiert auf den folgenden Erkentnissen, die durch die vorangegangenen Erfahrungen mit der Übernahme von Daten über Schweizer Kulturerbe-Institutionen erzielt wurden:

\begin{itemize}
\item Den fehlenden Überblick über Projekte und Daten zum Thema Kulturerbe auf Wikidata


\item Die Hürde für Mitwirkende den Prozess der Dateneingabe eigenständig zu bewältigen


\item Die Ausschöpfung des Potentials von Wikidata zum Thema Kulturerbe zur Verbesserung der Koordinierung, Dokumentation und Einbindung relevanter Partnern


\end{itemize}

Das Ziel dieses Projektes besteht in der Koordinierung, Erleichterung und Förderung der Aufnahme von Daten über das Kulturerbe in Wikidata, die Bereinigung und Verbesserung der Daten zu vereinfachen und die Nutzung nicht nur in Wikipedia und Schwesterprojekten, sondern auch darüber hinaus zu fördern. \autocite{noauthor_wikidatawikiproject_nodate}


\printbibliography[title={Literaturverzeichnis}]
\end{document}
