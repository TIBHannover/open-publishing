\documentclass{article}

\usepackage{hyperref}
\begin{document}

\title{Über das Buch "Kulturerbe kontextualisieren mit Wikidata"}

\maketitle





Dieses Buch stellt kommentierte Ressourcen zusammen, die den Einstieg in Wikidata erleichtern sollen. Konkreter Anwendungsbezug ist das Thema Kulturerbe, insbesondere für Bürger(wissenschaftler). 


Das Buch ist in vier Kapitel unterteilt, von denen die ersten beiden den inhaltlichen Kontext herstellen und die anderen beiden solche Quellen beschreiben, in denen Nutzer konkrete Hilfen zum Arbeiten in und mit Wikidata finden.


Pro Kapitel werden mindestens vier Quellen aufgelistet - mit jeweils einer kurzen Zusammenfassung zum Inhalt, mindestens drei Schlagworten und dem Verweis auf die (open access verfügbare) Ressource. [Bitte pro Kapitel mindestens 1 Bild ergänzen, z.B. aus einer der Quellen, sofern CC-BY oder CC-0). Die Ressourcen werden vorab in einer \href{https://www.zotero.org/groups/1838445/generation_r/collections/DND4FSHT}{kollaborativ kuratierten Bibliographie} (bitte \href{https://www.zotero.org/groups/1838445/generation_r}{hier }beitreten) gesammelt und mit dem zusätzlichen Tag "Wiki4Culture" versehen. Die Referenzen werden jeweils im Zitierstil Cicago 17th, full note aus Zotero exportiert.\emph{ - diese Sätze später wieder entfernen]}


Das Buch wurde als Teil der Lehrveranstaltung Open Knowledge im SoSe 2022 von Studierenden verfasst. Weitere Informationen zur Veranstaltung und weiteren Arbeiten im inhaltlichen Zusammenhang mit diesem Buch finden sich auf der zugehörigen \href{https://de.wikiversity.org/wiki/OpenKnowledge22}{Seite auf Wikiversity}.


add DOI, License, ..

\end{document}
