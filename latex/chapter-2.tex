\documentclass{article}

                
\usepackage[backend=biber,hyperref=false,citestyle=authoryear,bibstyle=authoryear]{biblatex}
                
\bibliography{bibliography}
            
\begin{document}

\title{Wikidata - Enabler von Citizen Science?}

\maketitle





Bürgerwissenschaften. [...] Bürgerwissenschaften im Kontext Kulturerbe [...] \autocite{ridge_collective_2021} geben einen aktuellen Leitfaden zur Umsetzung digital unterstützter Partizipationsprojekte für Einrichtungen des kulturellen Erbes. (Es werden zudem Projekte und Plattformen vorgestellt, z.B. https://www.zooniverse.org/, die weltweit größte Plattform für die Beteiligung von Bürgern bei Forschungsprojekten.) (aktuell: https://inos-project.eu/2022/05/19/cultural-heritage-threats-and-the-role-of-citizen-engagement-inside-and-outside-universities/)


Bekanntestes Beispiel Wikipedia: Bürger tragen zu wiss. Themen bei. [...] Funktionsweise der Wikimedia Community [...]. [mögliche Inhalte: Wikiversum und das Verständnis digitaler Allmende die von allen beschrieben werden kann // offen, leicht beitragen, … // was sind das für Leute die in ihrer Freizeit hier mitmachen? // wie sehen Events wie Editathons aus? // .. // Wikidata projects, wie organisiert sich die Community selbst?]


Wikidata als Hub für strukturiertes Wissen. Aktivstes WM Project aktuell. [...] Die Wikidata Interfaces ermöglichen es, dass jedermann mehrsprachige strukturierte Informationen zu seinen jeweiligen Interessengebieten beitragen kann, genauso wie schon vorher das Beitragen von Informationen als Fließtext in der jeweiligen Sprache in unstrukturierter Form bei Wikipedia möglich war und ist. Die jeweiligen Communities organisieren sich in WD in sogenannten WikiProjects. [...] Funktioniert [gut/schlecht/...] Wo sind die Hürden? [...]





\printbibliography[title={Literaturverzeichnis}]
\end{document}
