\documentclass{article}

\usepackage{caption}
                
\usepackage[backend=biber,hyperref=false,citestyle=authoryear,bibstyle=authoryear]{biblatex}
                
\bibliography{bibliography}
            
\usepackage{graphicx}
                
\usepackage{calc}
                
\newlength{\imgwidth}
                
\newcommand\scaledgraphics[2]{%
                
\settowidth{\imgwidth}{\includegraphics{#1}}%
                
\setlength{\imgwidth}{\minof{\imgwidth}{#2\textwidth}}%
                
\includegraphics[width=\imgwidth,height=\textheight,keepaspectratio]{#1}%
                
}
            
\begin{document}

\title{Wikidata - erste Schritte}

\maketitle





In diesem Kapitel werden vier Quellen zu der Wissensdatenbank Wikidata vorgestellt und beschrieben.


\subsection{Handbuch Wikidata}\label{H2539699}



Im Handbuch Wikidata \autocite{solohub_handbuch_2021} geben Studierende eine Einführung in das Wikimediaprojekt Wikidata. Dabei wird es eine ausführliche Erläuterung mit Screenshots geben, welche zeigen, wie die Suche nach Objekten in Wikidata erfolgt und wie ein neues Datenobjekt erstellt werden kann. Außerdem wird auf die Funktionen (Items, Items anlegen etc.) eingegangen und darüber hinaus werden Statements/Aussagen (vertieft) thematisiert. Das Handbuch Wikidata kann für Studierende sehr hilfreich sein; sowohl für diejenigen, die sich mit Wikidata auskennen, als auch für diejenigen, die sich damit zum ersten Mal beschäftigen, da die Inhalte auf eine simple Art erklärt werden. 

\begin{figure}
\scaledgraphics{c974b28d-908a-42e2-9445-5efb50e2817b.png}{1}
\caption*{Abbildung 1: "lebendes" Handbuch Wikidata}\label{F44920591}
\end{figure}





\subsection{Wikidata:Einführung}\label{H8838465}



Ähnlich wie die erste Quelle beschreibt die Informationsseite Wikidata:Einführung \autocite{noauthor_wikidataeinfuhrung_nodate} vertiefter die Bedeutung und Funktionalität der Wikidata. Diese Inhalte werden nicht nur als Text zusammengefasst, sondern auch mithilfe der Abbildungen und einer Tabelle mit Beispielen visualisiert. Diese Quelle enthält mehrere interne Verlinkungen wie zum Beispiel Help:Objekt \autocite{wikidata_helpobjekte_nodate} oder Help:Aussagen \autocite{wikidata_helpaussagen_nodate}. Die erste Seite erläutert über welche Kriterien ein Datenobjekt idealerweise verfügt, was vor und während der Objektserstellung beachtet werden soll und wie sich das Objekt löschen lässt. Die Help:Aussagen informiert den Leser, wie ein Statement hinzugefügt werden kann, welche Voraussetzungen die Werte erfüllen müssen und wie die Reihenfolge bei den ganzen Aussagen und einzelnen Eigenschaften funktioniert. In dieser Seite stehen ebenfalls mehrere Abbildungen zur Verfügung, die entweder die Bestandteile eines Statements an einem Beispiel verdeutlichen oder schrittweise die Erstellung einer Aussage erklären.

\begin{figure}
\scaledgraphics{2cea247d-8f76-4b1b-849c-7d7a9ad677ae.png}{1}
\caption*{Abbildung 2: Struktur eines Objekts aus "Wikidata:Einführung"}\label{F31998531}
\end{figure}





\subsection{Wikidata-Touren}\label{H784515}



Auf der Seite Wikidata-Touren \autocite{wikidata_wikidatatours_nodate} bekommen die Interessenten erste praktische Einblicke in die Nutzung von Wikidata. Anhand dieser Quelle können die Nutzer darüber hinaus die Grundlagen und die Aktivitäten von Wikidata erlernen; zum Beispiel wie Wikidata funktioniert, wie die Datenobjekte strukturiert sind und welche Bestandteile die Struktur dieser Objekte hat. Die so genannten Touren bestehen aus kleinen Aufgaben mit einem Hilfetext, die es nicht nur für die Grundlagen (Objekte und Aussagen), sondern auch für die fortgeschrittene Nutzung (Anbindung von Koordinaten oder Bildern) gibt.


\subsection{Screencast zur Datenanlegung in Wikidata}\label{H4699156}



In einem fünfminütigen Screencast „Daten in Wikidata anlegen“ \autocite{sidik_tool_2022} wird anhand eines Beispiels gezeigt, wie ein Datenobjekt in Wikidata erstellt wird. Zunächst werden allgemeine Informationen zu Wikidata genannt; darauffolgend wird erklärt, was vor der Anlegung des Datenobjekts beachtet wird und wie die Suchfunktion des Portals funktioniert; demnächst wird ein neues Datenobjekt erstellt, in dem seine Beschreibung, Bezeichnung und Aliasse auf Deutsch definiert werden; wenn die neue Seite erstellt wird, werden die oben genannten Attribute dieses Objekts auf Englisch in einer Tabelle ausgefüllt; anschließend wird dem Datenobjekt eine Aussage hinzugefügt.


\printbibliography[title={Literaturverzeichnis}]
\end{document}
